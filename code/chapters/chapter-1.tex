\chapter{Introduction}

\section{The Chaos}

It all began in the early 16th century. Mathematicians were being haunted—not by demons, but by a single, innocent-looking equation. The world was changing, ideas were blossoming, and yet, one question kept echoing through the minds of the greats:

\emph{How do we solve an equation like $x^2 + 1 = 0$?}

Girolamo Cardano, one of the pioneering minds of the Renaissance, while working on the theory of polynomials, stumbled upon this ghostly riddle. The question sounded deceptively simple:

\begin{center}
\emph{"Does there exist a number which, when squared, gives the negative of one?"}
\end{center}

But why was the curiosity so deeply tied to \emph{unity}, i.e., $1$? Simple—because if we could tame $x^2 + 1 = 0$, we could solve $x^2 + a = 0$ for any real $a$. Mathematicians knew that if some mysterious number $i$ satisfied

\[
i^2 = -1,
\]

then automatically,

\[
(ix)^2 = -x^2 \quad \text{for any } x \in \mathbb{R}.
\]

In short, solve it once for $1$, and you solve it for everything.

But this wasn’t just a puzzle—it was a crisis. The number $i$ didn’t live in the world of real numbers. It belonged to a realm yet undefined. Mathematicians at the time were bound to the real line. They didn’t know there was a whole new dimension waiting for them.

And then, in 1572, a spark appeared.

\subsection*{Rafael Bombelli: The Brave Alchemist of Numbers}

Bombelli, a fearless Italian mathematician, trusted algebra like a friend. He did the unthinkable:

\[
\sqrt{-1} \cdot \sqrt{-1} = -1.
\]

It seems so obvious now—but back then, this was heresy! Why hadn’t anyone else written it? Because mathematics demands not courage, but \emph{rigour}. The square root operation was never defined for negative numbers, and jumping ahead could mean logical disaster. Yet Bombelli took that leap—not carelessly, but carefully—opening the door to a whole new world.

\subsection*{Wessel and Argand: The Visionaries}

Later, in the late 18th and early 19th centuries, two legends—Caspar Wessel (1797) and Jean-Robert Argand (1806)—entered the scene. They didn’t just accept the existence of $i$; they visualized it.

They realized something profound:

- If real numbers like $1$, $2$, $\sqrt{2}$ exist,
- And imaginary numbers like $i$, $2i$, $\sqrt{2}i$ exist,
- Then numbers of the form $a + bi$ must exist too.

These became known as **Complex Numbers**, denoted by $\mathbb{C}$.

And here came the marvel—they saw multiplication not just as arithmetic, but as **scaling and rotation**.

\subsection*{The Magic of $i$}

They studied this bizarre identity:

\[
i \cdot i = -1.
\]

Geometrically, this meant: starting at $1$, multiply by $i$ once—and you turn 90° counterclockwise. Multiply again—and you've made a half-turn. You’ve reached $-1$.

\begin{center}
\begin{tikzpicture}[scale=2]

  % Draw the semicircle
  \draw[thick, ->, postaction={decorate},
        decoration={markings, mark=at position 0.6 with {\arrow{>}}}]
    (1,0) arc (0:180:1);

  % Draw the diameter (optional, for clarity)
  \draw[dashed] (-1,0) -- (1,0);

  % Points on ends
  \filldraw (-1,0) circle (0.03);
  \filldraw (1,0) circle (0.03);

  % Labels
  \node[below] at (-1,0) {$-1$};
  \node[below] at (1,0) {$1$};

\end{tikzpicture}

\textit{Multiplying by $i$ twice is a rotation by $\pi$ radians.}
\end{center}

So naturally, they asked:

\begin{quote}
“If multiplying by $i$ twice rotates by $\pi$, then multiplying once must rotate by $\frac{\pi}{2}$, right?”
\end{quote}

This thought ignited the idea of a **two-dimensional number line**.

\subsection*{The Birth of the Argand Plane}

\begin{center}
\begin{tikzpicture}[scale=1.5, >=stealth]

  % Axes
  \draw[->] (-2.2,0) -- (2.2,0) node[right] {$\Re$};
  \draw[->] (0,-2.2) -- (0,2.2) node[above] {$\Im$};

  % Points
  \filldraw[black] (1,0) circle (0.05) node[below right] {$1$};
  \filldraw[black] (-1,0) circle (0.05) node[below left] {$-1$};
  \filldraw[black] (0,1) circle (0.05) node[above right] {$i$};
  \filldraw[black] (0,-1) circle (0.05) node[below right] {$-i$};
  \filldraw[black] (0,0) circle (0.04) node[below left] {$0$};

\end{tikzpicture}
\end{center}

Every complex number $z = a + ib$ is now just a point in this plane—called the **Argand Plane**. And every such number has a natural position vector from the origin.

This laid the foundation for the majestic study of complex numbers, and later, complex-valued functions—what we today call **Complex Analysis**.

Oh, and by the way, that’s why we write $i$, $2i$, $3i$, etc., on the vertical axis not $\sqrt{-1}$, $2\sqrt{-1}$... because $i$ was designed to follow a geometric thought process. It fit the rotation; it respected the structure.

---

\section{Operations on Complex Numbers}

Now let’s explore the operations.

Let
\[
z_1 = a_1 + b_1 i, \quad z_2 = a_2 + b_2 i
\]
Then addition is simple:
\[
z_1 + z_2 = (a_1 + a_2) + (b_1 + b_2)i
\]

Multiplication is where the magic deepens:
\begin{align*}
z_1 \cdot z_2 &= (a_1 + b_1 i)(a_2 + b_2 i) \\
&= a_1 a_2 + a_1 b_2 i + b_1 a_2 i + b_1 b_2 i^2 \\
&= (a_1 a_2 - b_1 b_2) + (a_1 b_2 + b_1 a_2)i
\end{align*}

And division? Just as neat:
\begin{align*}
\frac{z_1}{z_2} &= \frac{a_1 + b_1 i}{a_2 + b_2 i} \cdot \frac{a_2 - b_2 i}{a_2 - b_2 i} \\
&= \frac{(a_1 a_2 + b_1 b_2) + (b_1 a_2 - a_1 b_2)i}{a_2^2 + b_2^2} \\
&= \frac{a_1 a_2 + b_1 b_2}{a_2^2 + b_2^2} + \frac{b_1 a_2 - a_1 b_2}{a_2^2 + b_2^2}i
\end{align*}

This tells us that the set $\mathbb{C}$ is **closed** under addition, subtraction, multiplication, and division. This is called the **closure property**—and it’s one of the first signs of $\mathbb{C}$'s algebraic power.

Similarly, we know that
\begin{align*}
z_1 \cdot z_2 &= (a_1 a_2 - b_1 b_2) + (a_1 b_2 + b_1 a_2)i
\end{align*}

Now let us reverse the order:
\begin{align*}
z_2 \cdot z_1 &= (a_2 a_1 - b_2 b_1) + (a_2 b_1 + b_2 a_1)i
\end{align*}

Similarly, let us check for addition:
\[
z_1 + z_2 = (a_1 + a_2) + (b_1 + b_2)i
\]
And after reversing the order:
\[
z_2 + z_1 = (a_2 + a_1) + (b_2 + b_1)i
\]

Since both \emph{addition} and \emph{multiplication} are commutative for complex numbers, we conclude that the set $\mathbb{C}$ is commutative under both operations.

Furthermore, one can verify that $\mathbb{C}$ is \emph{associative}, meaning the grouping of operations does not affect the result. This follows from the fact that $\mathbb{R}$ is associative under addition and multiplication.

Let us consider three complex numbers:
\begin{align*}
z_1 &= a_1 + b_1 i \\
z_2 &= a_2 + b_2 i \\
z_3 &= a_3 + b_3 i
\end{align*}

Now compute:
\begin{align*}
z_1 (z_2 + z_3) &= (a_1 + b_1 i)((a_2 + a_3) + (b_2 + b_3)i) \\
&= a_1(a_2 + a_3) - b_1(b_2 + b_3) + [a_1(b_2 + b_3) + b_1(a_2 + a_3)]i \\
&= z_1 z_2 + z_1 z_3
\end{align*}

Thus,
\[
z_1(z_2 + z_3) = z_1 z_2 + z_1 z_3
\]
This is called the \emph{left distributive property}. Similarly, we can verify:
\[
(z_2 + z_3)z_1 = z_2 z_1 + z_3 z_1
\]
which is called the \emph{right distributive property}. Hence, $\mathbb{C}$ is distributive under multiplication.

Therefore, the set $\mathbb{C}$ satisfies closure, associativity, distributivity, and commutativity—just like the set $\mathbb{R}$.

Now, returning to operations on complex numbers: we already understand that any $z \in \mathbb{C}$ can be visualized as a position vector in the Argand plane. Thus, it must have both a magnitude and an angle.

\begin{center}
\begin{tikzpicture}[scale=2, >=stealth]
  % Axes
  \draw[->, thick] (-0.5,0) -- (3.2,0) node[right] {$\Re$};
  \draw[->, thick] (0,-0.5) -- (0,2.5) node[above] {$\Im$};

  % Coordinates
  \coordinate (O) at (0,0);
  \coordinate (Z) at (2,1.5);

  % Point z = a + bi
  \filldraw[black] (Z) circle (0.05) node[above right] {$z = a + bi$};

  % Dotted lines
  \draw[dashed, black] (Z) -- (2,0) node[below] {$a$};
  \draw[dashed, black] (Z) -- (0,1.5) node[left] {$bi$};

  % Position vector
  \draw[->, thick, black] (O) -- (Z);

  % Origin
  \filldraw[black] (O) circle (0.04) node[below left] {$0$};

  % Angle theta arc
  \draw[->, thick] (1.2,0) arc[start angle=0, end angle=36.87, radius=1.2];
  \node at (0.8,0.25) {$\theta$};
\end{tikzpicture}
\end{center}

Let the real part of $z$ be denoted by $\operatorname{Re}(z)$ and the imaginary part by $\operatorname{Im}(z)$. Let $R$ denote the magnitude of the vector $z$. Then using basic trigonometry:

\begin{align}
\sin \theta &= \frac{\operatorname{Im}(z)}{R} \tag{1.1} \\
\cos \theta &= \frac{\operatorname{Re}(z)}{R} \tag{1.2}
\end{align}

Since
\[
z = \operatorname{Re}(z) + \operatorname{Im}(z) \cdot i,
\]
multiplying both sides of (1.1) and (1.2) by $R$ gives:
\begin{align}
z = R\left( \cos \theta + i \sin \theta \right) \tag{1.3}
\end{align}

Equation (1.3) is the polar form of a complex number, a new and insightful representation of $z \in \mathbb{C}$.
